\section{Plan und Todo}
\frame{
\frametitle{Plan und Todo}
TODO:
\begin{itemize}
  \item Installation (nach kurzer Einleitung, erwähnen: versch. Distros existieren etc.) durchführen
  \item einleitende Worte zu KDE
  \item Softwarestruktur in Linux + Paketmanager, zu Yast hinführen (-> exemplarisch Roberts Repo hinzufügen)
  \item generell: coole Programme vorstellen (kurz jeweils) (Bsp.: ssh (+scp,sshfs,...); kile(?); kdevelop; mathe-programme; bash(?); git; yakuake; convert
\end{itemize}
}

\section{SSH}
\frame{
\frametitle{Entferntes Arbeiten: ssh}
\begin{itemize}
  \item man kann bspw. Programme auf den Uni-Rechnern laufen lassen, die man selbst nicht installiert hat (z.B. Maple o.ä.)
  \item oder: Dateien einfach von Pool-Rechnern auf seinen PC kopieren, wenn man zu Hause weiter arbeiten will
\end{itemize}
}

\frame{
\frametitle{Wie geht das?}
Anmelden in Konsole:\\
\begin{center}
  \texttt{ssh user@pool1.physik.hu-berlin.de}
\end{center}
Dateien kopieren:\\
% \begin{center}
  \texttt{scp foo.bar user@pool1.physik.hu-berlin.de:path}\\
  \texttt{scp user@pool1.physik.hu-berlin.de:path/file.txt .}
% \end{center}
}

\frame{
\frametitle{Sicherheitsaspekte}
\begin{itemize}
  \item gut: Authentifizierung mit verschlüsseltem Key
  \item Key erzeugen:\\
        \begin{center}
          \texttt{ssh-keygen -b 1024 -t dsa}		% besser rsa?
        \end{center}
  \item Public Key nach $\sim$/.ssh kopieren und an Datei $\sim$/.ssh/authorized\_keys anhängen
  \item Rechte anpassen:\\
        \begin{center}
          \texttt{chmod 600 $\sim$/.ssh/authorized\_keys}
        \end{center}
  \item schließlich: ssh-agent verwenden, um Passwort nur einmal eingeben zu müssen (pro Sitzung):
        \begin{center}
          \texttt{ssh-add}
        \end{center}
\end{itemize}
}

\section{convert}
\frame{
\frametitle{Bildkonvertierung mit \texttt{convert}}
um Bilder zu konvertieren (z.B. jpg $\rightarrow$ png):
\begin{center}
  \texttt{convert a.jpg b.png}
\end{center}

}