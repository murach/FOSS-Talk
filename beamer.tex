\documentclass[usepdftitle=false,ucs]{beamer}
% \usepackage{beamerthemesplit}
\usepackage{graphicx,graphics}
\usepackage{wrapfig}
% \usepackage[ansinew]{inputenc}
% \usepackage{ucs}
\usepackage{lmodern}
\usepackage[T1]{fontenc}
\usepackage{mathptmx,courier,textcomp,booktabs}
\usepackage[scaled]{helvet}
\usepackage[ngerman]{babel}
\usepackage[utf8x]{inputenc}
\usepackage{xspace}
\usepackage{tikz}
\usetikzlibrary{decorations.pathreplacing}
\usepackage{framed, color}
\setbeamercovered{transparent}
\usetheme{CambridgeUS}
% \usetheme{Warsaw}
\usecolortheme{beaver}
\usepackage{siunitx}
% \usepackage{cooltooltips}
\beamertemplatenavigationsymbolsempty
% \setbeamertemplate{footline}[frame number]
\usepackage{listings}
\lstset{language=C++,
%   commentstyle=\itshape\color{darkgreen},
%   commentstyle=\color{darkgreen},
%   keywordstyle=\bfseries, %\color{darkblue},
%   stringstyle=\color{darkred},
%   basicstyle=\ttfamily\scriptsize,
  morekeywords={TH1F,TLorentzVector,TVector3,vector,TFile,TFitResultPtr,TF1,\
                TGraph,TH1,TObject,TCanvas,string,Double_t,TGraphErrors},
  basicstyle=\scriptsize,
  numbers=left,
  numberstyle=\tiny,%\color{gray},
  stepnumber=1,
  tabsize=4,
  showspaces=false,
  showstringspaces=false,
  breaklines=true,
  frame=lrtb,
  captionpos=b,
  extendedchars=true,
  inputencoding=utf8,
%   backgroundcolor=\color{ltgray}
}
\sisetup{
	seperr		=	true,
	trapambigerr	=	true,
	openerr		=	(,
	closeerr	=	),
	expproduct	=	cdot,
	padnumber	=	both,
	stickyper	=	true,
	per		=	reciprocal,
	trapambigfrac	=	true,
	repeatunits	=	false,
	openfrac	=	(,
	closefrac	=	),
	prefixsymbolic 	=	true,
	prefixproduct	=	cdot,
	decimalsymbol	=	comma,
% 	decimalsymbol	=	,
    tabnumalign     =       left,
    tabtextalign    =       left
%   separate-uncertainty = true,
%   multi-part-units = brackets,
%   open-bracket = (,
%   close-bracket = ),
%   exponent-product = \cdot,
%   add-decimal-zero = true,
%   add-integer-zero = true,
%   per-mode         =   reciprocal,
%   prefixes-as-symbols = true,
%   output-decimal-marker = {,},
%   table-figures-uncertainty = 1
}

\definecolor{ellipsegreen}{rgb}{0,0.55,0.3}
\setbeamercovered{invisible}		% making pause/uncover possible for images

\newcommand{\beginbackup}{
 \setbeamertemplate{footline}{
  \leavevmode%
  \hbox{%
  \begin{beamercolorbox}[wd=.333333\paperwidth,ht=2.25ex,dp=1ex,center]{author in head/foot}%
    \usebeamerfont{author in head/foot}\insertshortauthor~~(\insertshortinstitute)
  \end{beamercolorbox}%
  \begin{beamercolorbox}[wd=.333333\paperwidth,ht=2.25ex,dp=1ex,center]{title in head/foot}%
    \usebeamerfont{title in head/foot}\insertshorttitle
  \end{beamercolorbox}%
  \begin{beamercolorbox}[wd=.333333\paperwidth,ht=2.25ex,dp=1ex,right]{date in head/foot}%
    \usebeamerfont{date in head/foot}\insertshortdate{}\hspace*{2em}
    \insertframenumber{} \hspace*{2ex} % hier hat's sich geändert
  \end{beamercolorbox}}%
  \vskip0pt%
 }
   \newcounter{framenumbervorappendix}
   \setcounter{framenumbervorappendix}{\value{framenumber}}
}
\newcommand{\backupend}{
   \addtocounter{framenumbervorappendix}{-\value{framenumber}}
   \addtocounter{framenumber}{\value{framenumbervorappendix}} 
}


\title[KDE on Linux]{How to work with Linux and KDE as a scientist}
\author{Thomas Murach \and Robert Riemann}
\institute[HU Berlin]{HU Berlin}
\date[HU, 09.03.2012]{09.03.2012}

\hypersetup
{
	pdfauthor	=	{Thomas Murach},
	pdftitle	=	{How to work with Linux and KDE as a scientist}
}


\begin{document}

% \renewcommand{\inserttotalframenumber}{17}

\logo{\includegraphics[height=1cm]{siegel.pdf}}
% \frame{
% \titlepage
% }

\maketitle
% 
% \frame{
% \frametitle{Gliederung}
% \tableofcontents
% }
% \section{Progress}
\section{Plan und Todo}
\frame{
\frametitle{Plan und Todo}
TODO:
\begin{itemize}
  \item Installation (nach kurzer Einleitung, erwähnen: versch. Distros existieren etc.) durchführen
  \item einleitende Worte zu KDE
  \item Softwarestruktur in Linux + Paketmanager, zu Yast hinführen (-> exemplarisch Roberts Repo hinzufügen)
  \item generell: coole Programme vorstellen (kurz jeweils) (Bsp.: ssh (+scp,sshfs,...); kile(?); kdevelop; mathe-programme; bash(?); git; yakuake; convert
\end{itemize}
}

\section{SSH}
\frame{
\frametitle{Entferntes Arbeiten: ssh}
\begin{itemize}
  \item man kann bspw. Programme auf den Uni-Rechnern laufen lassen, die man selbst nicht installiert hat (z.B. Maple o.ä.)
  \item oder: Dateien einfach von Pool-Rechnern auf seinen PC kopieren, wenn man zu Hause weiter arbeiten will
\end{itemize}
}

\frame{
\frametitle{Wie geht das?}
Anmelden in Konsole:\\
\begin{center}
  \texttt{ssh user@pool1.physik.hu-berlin.de}
\end{center}
Dateien kopieren:\\
% \begin{center}
  \texttt{scp foo.bar user@pool1.physik.hu-berlin.de:path}\\
  \texttt{scp user@pool1.physik.hu-berlin.de:path/file.txt .}
% \end{center}
}

\frame{
\frametitle{Sicherheitsaspekte}
\begin{itemize}
  \item gut: Authentifizierung mit verschlüsseltem Key
  \item Key erzeugen:\\
        \begin{center}
          \texttt{ssh-keygen -b 1024 -t dsa}		% besser rsa?
        \end{center}
  \item Public Key nach $\sim$/.ssh kopieren und an Datei $\sim$/.ssh/authorized\_keys anhängen
  \item Rechte anpassen:\\
        \begin{center}
          \texttt{chmod 600 $\sim$/.ssh/authorized\_keys}
        \end{center}
  \item schließlich: ssh-agent verwenden, um Passwort nur einmal eingeben zu müssen (pro Sitzung):
        \begin{center}
          \texttt{ssh-add}
        \end{center}
\end{itemize}
}

\section{convert}
\frame{
\frametitle{Bildkonvertierung mit \texttt{convert}}
um Bilder zu konvertieren (z.B. jpg $\rightarrow$ png):
\begin{center}
  \texttt{convert a.jpg b.png}
\end{center}
(benötigt das Paket „ImageMagick“)
}

\section{Vektorgraphiken konvertieren}
\frame{
\frametitle{Vektorgraphiken konvertieren}
Um Vektorgraphiken zu konvertieren, gibt es viele kleine Programme:
\texttt{ps2pdf} bzw. \texttt{pstopdf}, \texttt{epstopdf}, etc.
}


\end{document}